
\section{Tutorial}
\label{section:tutorial}
\setcounter{footnote}{0}

Here's a tutorial walk-through of some how to use \rscape. This should
suffice to get you started.

\subsection {Modes of \rscape}

\begin{tabular}{ll}
\multicolumn{2}{c}{\textbf{MSA input with annotated consensus secondary structure}} \\ 
 & \\ 
\textbf{R-scape}   & Reports all pairs which covariation scores have E-values smaller or equal to a target E-value.\\
\textbf{}          & Draws the given consensus structure annotated with the significantly covarying base pairs.\\
 & \\ 
\multicolumn{2}{c}{\textbf{MSA input without an annotated secondary structure}}  \\
 & \\ 
\textbf{R-scape}   & Reports all pairs which covariation scores have E-values smaller or equal to a target E-value.\\
\textbf{}          & Builds the best consensus structure that includes all significantly covarying pairs,\\
\textbf{}          & \hspace{5mm}\emph{the maximum-covariation consensus structure}.\\
\textbf{}          & Draws the \emph{maximum-covariation consensus structure} annotated with the significantly covarying base pairs.\\
 & \\ 
\end{tabular} \\
\\

In the Tutorial section, I'll show examples of running each \rscape,
using examples in the \ccode{tutorial/} subdirectory of the
distribution.


\subsection{Files used in the tutorial}

The subdirectory \prog{/tutorial} in the \rscape\ distribution contains the
files used in the tutorial. 

The tutorial provides several examples of RNA structural
alignments, all in Stockholm format:

\begin{sreitems}{\emprog{minifam.h3\{m,i,f,p\}}}
\item[\emprog{updated\_arisong.sto}] Structural alignment of the ciliate
  Arisong RNA. This alignment is an updated
  version of the one published in \citep{JungEddy11}.
\item[\emprog{ar14.sto}] Structural alignment of the $\alpha$-proteobacteria ncRNA ar14. This alignment is an updated version of the one
  published in \citep{delVal12}.
\item[\emprog{RF00001.sto}] Rfam v12.0 seed alignment of 5S rRNA. 
\item[\emprog{RF00005.sto}] Rfam v12.0 seed alignment of tRNA. 
\item[\emprog{RF00001.sto}] Rfam v12.0 seed alignment of YRNA. 
\end{sreitems}



\subsection{Outputs of running \rscape}

For each alignment file \prog{rna.sto}, \rscape\, produces the following output:

\begin{sreitems}{\emprog{rna.sorted.out}}
\item[\emprog{rna.out}] Tabular output with the significant pairs,
  with their score and E-value.
%
\item[\emprog{rna.sorted.out}] Tabular output sorted from highest to
  lowest E-value.
%
\item[\emprog{rna.roc}] Tabular output that provides statistics for 
each score value.
%
\item[\emprog{rna.sum}] Tabular output with a line summary statistics
  per alignment in the file.
%
\end{sreitems}

An Stockholm alignment file can include several alignments.  For each
individual alignment (\prog{n}) in the Stockholm file,
\rscape\ produces the following output:

\begin{sreitems}{\emprog{rna\_\{n\}.R2R.sto.\{pdf,svg\}}}
\item[\emprog{rna\_\{n\}.his}] A two column histogram file for the
  covariation scores.
%
\item[\emprog{rna\_\{n\}.his.ps}] Plot of the score histogram.
%
\item[\emprog{rna\_\{n\}.R2R.sto}] Stockholm file annotated by a
  modified version of the R2R program. This file includes the
  information necesary to draw the consensus structure, and to
  annotate the significantly covarying base pairs.
%
\item[\emprog{rna\_\{n\}.R2R.sto.\{pdf,svg\}}] Drawing of the
  \rscape-annotated consensus secondary structure.
%
\item[\emprog{rna\_\{n\}.dplot.\{ps,svg\}}] Dot plot of the consensus
  secondary structure annotated according to covariation.
%
\end{sreitems}

If run with the option \prog{\rscape\ --cyk}, for each alignment
(\prog{n}) it produces the following additional files describing the
maximal-covariation optimal secondary structure:

\begin{sreitems}{\emprog{rna\_\{n\}.cyk.R2R.sto.\{pdf,svg\}}}
\item[\emprog{rna\_\{n\}.cyk.his}]
%
\item[\emprog{rna\_\{n\}.cyk.his.\{ps.svg\}}]
%
\item[\emprog{rna\_\{n\}.cyk.R2R.sto}]
%
\item[\emprog{rna\_\{n\}.cyk.R2R.sto.\{pdf,svg\}}]
%
\item[\emprog{rna\_\{n\}.cyk.dplot.\{ps,svg\}}]
%
\end{sreitems}
These files are formatted identially to those for describing the given
consensus structure.



\subsection{Running \rscape\, on one alignment file}

\prog{bin/R-scape tuturial/updated\_arisong.sto}

\begin{sreoutput}
\end{sreoutput}








