
\section{Tutorial}
\label{section:tutorial}
\setcounter{footnote}{0}

Here's a tutorial walk-through of how to use \rscape. This should
suffice to get you started.

\subsection {Modes of \rscape}

For an input alignment, \rscape\ reports all pairs that have
covariation scores with E-values smaller than a target E-value.\\

\noindent
The E-values are calculated in one of two ways:

\begin{tabular}{ll}
\multicolumn{2}{l}{\textbf{A one-set statistical test:} \textit{default}} \\ 
 & \\ 
\textbf{}   & E-values are calculated assuming that all pairs are possible.\\
\textbf{}   & This is the default behaviour of \rscape.\\
 & \\ 
\multicolumn{2}{l}{\textbf{A two-set statistical test: } \prog{option -s}} \\ 
 & \\ 
\textbf{}   & If the alignment has associated a \emph{given structure}, \textbf{\prog{option -s}} performs two independent statistical tests: \\
\textbf{}   & one for the pairs included in the structure, a different one for all the remaining possible pairs.\\
\textbf{}   & It also draws the given consensus structure annotated with the significantly covarying base pairs.\\
 & \\ 
\end{tabular}

\subsection {Option --cyk}

After performing one of the two statistical tests, this option:\\

\begin{tabular}{ll}
\textbf{}   & Builds the best consensus structure that includes the largest possible number of significantly covarying pairs,\\
\textbf{}   & \hspace{5mm}\emph{the maximum-covariation optimal consensus structure}.\\
\textbf{}   & Draws the \emph{maximum-covariation optimal consensus structure} annotated with the significantly \\
\textbf{}   & \hspace{5mm}covarying base pairs.\\
\textbf{}   & It also returns the alignment in Stockholm format annotated with the max-cov optimal consensus structure.\\
 & \\ 
\end{tabular}


I'll show examples of running each mode, using examples in the
\ccode{tutorial/} subdirectory of the distribution.


\subsection{Files used in the tutorial}

The subdirectory \prog{/tutorial} in the \rscape\ distribution contains the
files used in the tutorial. 

The tutorial provides several examples of RNA structural
alignments, all in Stockholm format:

\begin{sreitems}{\emprog{updated\_Arisong.sto}}
\item[\emprog{updated\_Arisong.sto}] Structural alignment of the ciliate
  Arisong RNA. This alignment is an updated
  version of the one published in~\citep{JungEddy11}.
\item[\emprog{ar14.sto}] Structural alignment of the $\alpha$-proteobacteria ncRNA ar14. This alignment is an updated version of the one
  published in~\citep{delVal12}.
\item[\emprog{RF00005.sto}] Rfam v12.0~\citep{Nawrocki15} seed alignment of tRNA. 
\item[\emprog{RF00001-noss.sto}] Rfam v12.0 seed alignment of 5S rRNA, after removing the consensus secondary structure. 
\end{sreitems}


\subsection{Running \rscape\, on one alignment file}
To run \rscape\ with default parameters on alignment file
\prog{tutorial/updated\_Arisong.sto} use:\\

\user{bin/R-scape tutorial/updated\_Arisong.sto}\\

\noindent
The output is a list of the significantly covarying positions under the one-set test

\begin{sreoutput}
# R-scape :: RNA Structural Covariation Above Phylogenetic Expectation
# R-scape 0.8.1 (Jul 2018)
# Copyright (C) 2016 Howard Hughes Medical Institute.
# Freely distributed under the GNU General Public License (GPLv3).
# - - - - - - - - - - - - - - - - - - - - - - - - - - - - - - - - - - - -
# One-set statistical test (all pairs are tested as equivalent) 
#
# MSA updated_Arisong_1 nseq 95 (95) alen 65 (150) avgid 66.35 (64.97) nbpairs 20 (20)
#
# Method Target_E-val [cov_min,conv_max] [FP | TP True Found | Sen PPV F] 
# GTp    0.05         [-9.82,121.80]     [0 | 4 20 4 | 20.00 100.00 33.33] 
#
#       left_pos       right_pos        score           E-value
#------------------------------------------------------------------
*	      98	     106	121.80433	4.31981e-05
*	     122	     137	91.75573	0.00337167
*	      96	     108	89.46430	0.00470161
*	     120	     139	75.03790	0.0381599
\end{sreoutput}
A star ``*'' in the first column indicates that the pair is part of
the annotated structure in the \prog{updated\_Arisong.sto} file. A
blank indicates a pair that is not compatible with the structure. A
``$\sim$`` indicates an interaction not in the annotated structure but
compatible with it (none in this example).

The \prog{tutorial/updated\_Arisong.sto} has a proposed secondary
structure.  Instead of testing all pairs as equivalent, we may want to
test the significance of the given structure as a one set of pairs,
and independently that of the rest of all possible pairs.  In order to
do a two-set test use:\\

\user{bin/R-scape -s tutorial/updated\_Arisong.sto}\\

\noindent
The output is a list of the significantly covarying positions under the two-set test.

\begin{sreoutput}
# R-scape :: RNA Structural Covariation Above Phylogenetic Expectation
# R-scape 0.8.1 (Jul 2018)
# Copyright (C) 2016 Howard Hughes Medical Institute.
# Freely distributed under the GNU General Public License (GPLv3).
# - - - - - - - - - - - - - - - - - - - - - - - - - - - - - - - - - - - -
# Two-set statistical test (one test for annotated basepairs, another for all other pairs)
#
# Structure obtained from the msa
# MSA updated_Arisong_1 nseq 95 (95) alen 65 (150) avgid 66.35 (64.97) nbpairs 20 (20)
#
# Method Target_E-val [cov_min,conv_max] [FP | TP True Found | Sen PPV F] 
# GTp    0.05         [-9.82,121.80]     [0 | 12 20 12 | 60.00 100.00 75.00] 
#
#       left_pos       right_pos        score           E-value
#------------------------------------------------------------------
*	      98	     106	121.80433	4.15366e-07
*	     122	     137	91.75573	3.24199e-05
*	      96	     108	89.46430	4.52078e-05
*	     120	     139	75.03790	0.000366922
*	     119	     140	58.25176	0.00410697
*	     121	     138	57.96915	0.00421236
*	      94	     110	56.91065	0.00490376
*	     124	     134	55.84207	0.00570817
*	     123	     135	55.50367	0.0060045
*	      99	     105	53.86423	0.00773238
*	      97	     107	44.72409	0.0279485
*	     115	     144	41.87792	0.0427906
\end{sreoutput}
The scores of the pairs are identical to those in the one-set
test. The E-values have changed relative to those of the one-set test.

\subsection{Default parameters}

Default parameters are:

\begin{sreitems}{\emprog{Pairwise percent identity:}}
\item[\emprog{Target E-value:}]default is 0.05. \rscape\, reports
  pairs which covariation score has E-value smaller or equal to the
  target value.  The target E-value can be changed with option
  \emprog{-E <x>}, $x >= 0$.

\item[\emprog{Sequence weighting:}]Sequences are weighted according to
  the Gerstein/Sonnhammer/Chothia (GSC)
  algorithm~\citep{Gerstein94}. This algorithm is time consuming. For
  alignments with more than 1000 sequences, we use the faster
  position-based weighting algorithm~\citep{Henikoff94b}. Both
  weighting algorithms are implemented as part of the easel library.

\item[\emprog{Gaps in columns:}]Columns with more than 50\% gaps are
  removed. The gap threshold for removing columns can be modified
   using option \emprog{--gapthresh <x>} , $0<x<=1$.

 \item[\emprog{Covariation statistic:}]The default covariation statistic
   is the average product corrected G-Test (equivalent to option
   \emprog{--GTp}).

 \item[\emprog{Covariation Class:}]\rscape\ uses the 16 component
   covariation statistic (C16), unless the number of sequences in the
   alignment is $\leq$ 8 or the length of the alignment is $\leq$ 50,
   in which case it uses the two-class covariation statistic (C2). A
   particular covariation class can be selected using either
   \emprog{--C16} or \emprog{--C2}.

   The threshold for the minimum number of sequences can be changed
   with option \prog{--nseqthresh <n>}.  The threshold for the minimum
   alignment length can be changed with option \prog{--alenthresh <n>}.

 \item[\emprog{Null alignments:}]In order to estimate E-values,
   \rscape\ produces 20 null alignments, unless the product of the
   number of sequences by the length of the alignment $<$ 10,000 in
   which case the number of null alignments is 50; or $<$ 1,000 in
   which case it is 100. The number of null alignments can be
   controlled with option \emprog{--nshuffle <n>}.
 \end{sreitems}

 A full list of the \rscape\ options is found by using

 \user{\rscape\ -h}

 
 
