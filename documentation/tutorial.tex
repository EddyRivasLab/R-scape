
\section{Tutorial}
\label{section:tutorial}
\setcounter{footnote}{0}

Here's a tutorial walk-through of some how to use \rscape. This should
suffice to get you started.

\subsection {Modes of \rscape}

\begin{tabular}{ll}
\multicolumn{2}{c}{\textbf{MSA input with annotated consensus secondary structure}} \\ 
 & \\ 
\textbf{R-scape}   & Reports all pairs which covariation scores have E-values smaller or equal to a target E-value.\\
\textbf{}          & Draws the given consensus structure annotated with the significantly covarying base pairs.\\
 & \\ 
\multicolumn{2}{c}{\textbf{MSA input without an annotated secondary structure}}  \\
 & \\ 
\textbf{R-scape}   & Reports all pairs which covariation scores have E-values smaller or equal to a target E-value.\\
\textbf{}          & Builds the best consensus structure that includes all significantly covarying pairs,\\
\textbf{}          & \hspace{5mm}\emph{the maximum-covariation consensus structure}.\\
\textbf{}          & Draws the \emph{maximum-covariation consensus structure} annotated with the significantly covarying base pairs.\\
 & \\ 
\end{tabular} \\
\\

In the Tutorial section, I'll show examples of running each \rscape,
using examples in the \ccode{tutorial/} subdirectory of the
distribution.


\subsection{Files used in the tutorial}

The subdirectory \prog{/tutorial} in the \rscape\ distribution contains the
files used in the tutorial. 

The tutorial provides several examples of RNA structural
alignments, all in Stockholm format:

\begin{sreitems}{\emprog{minifam.h3\{m,i,f,p\}}}
\item[\emprog{updated\_arisong.sto}] Structural alignment of the ciliate
  Arisong RNA. This alignment is an updated
  version of the one published in \citep{JungEddy11}.
\item[\emprog{ar14.sto}] Structural alignment of the $\alpha$-proteobacteria ncRNA ar14. This alignment is an updated version of the one
  published in \citep{delVal12}.
\item[\emprog{RF00005.sto}] Rfam v12.0 seed alignment of tRNA. 
\item[\emprog{RF00001.sto}] Rfam v12.0 seed alignment of YRNA. 
\item[\emprog{RF00001-noss.sto}] Rfam v12.0 seed alignment of 5S rRNA, after removing the consensus secondary structure. 
\end{sreitems}


\subsection{Running \rscape\, on one alignment file}
To run \rscape\ with default parameters on alignment file
\prog{tutorial/updated\_arisong.sto} use:\\

\user{bin/R-scape tutorial/updated\_arisong.sto}\\

Default parameters are:\\
\begin{sreitems}{\emprog{Avg. percent identity:}}
\item[\emprog{Target E-value:}]Reports pairs which covariation
scores have E-values smaller or equal to 0.05. The target E-value can be changed
  with option \emprog{-E <x>}.

\item[\emprog{Avg. percent identity:}]sequences with more than
  97\% similarity to each other are removed.  The maximum average
  percentage identity in the alignment can be changed with option
  \emprog{-I <x>}.

\item[\emprog{Gaps in columns}]columns with more than 50\% gaps are
  removed. The gap threshold for removing columns can be modified
  using option \emprog{--gapthresh <x>}

\item[\emprog{Covariation statistic}]covariation statistic in the
  product-average corrected G-Test (\emprog{--GTp}).

\item[\emprog{Covariation Class}]uses the 16 compoment covariation
  statistic (C16), unless the number of sequences in the alignment is
  $\leq$ 8 or the length of the alignment is $\leq$ 50, in which case
  it uses the two-class covariation statistic (C2). A particular
  covariation class can be selected using either \emprog{--C16} or
  \emprog{--C2}.

  The threshold for the minimun number of sequences can be changed
  with option \prog{--nseqthresh <n>}.  The threshold for the minimun
  alignment length can be changed with option \prog{--alenthresh <n>}.

\item[\emprog{Null alignments:}]produces 20 null
  alignments, unless the product of the number of sequences by the
  length of the alignment $<$ 10,000 in which case the number of null
  alignments is 50; or $<$ 1,000 in which case it is 100. The
  number of null alignments can be controlled with option
  \emprog{--nshuffle <n>}.
\end{sreitems}


