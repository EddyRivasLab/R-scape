\section{Introduction}
\setcounter{footnote}{0}

\rscape\ (RNA Significant Covariation Above Phylogenetic Expectation)
is a program that given a multiple sequence alignment (MSA) of RNA
sequences, finds the pairs of positions that show a pattern of
significant covariation. Each covariation score has an E-value
associated to it. E-values are determined using a null model of
covariation due to phylogeny but independent of any structural
constraints. 

\subsection{How to avoid reading this manual}

\begin{itemize}
\item Follow the quick installation instructions on page
      \pageref{section:installation}. 
\item Go to the tutorial section on page
\pageref{section:tutorial}, which walks you through some examples of
using \rscape\ on real data.
\end{itemize}

Everything else, you can read later.



\subsection{How do I cite \rscape?}

Rivas, E. \textit{et al.}, \textit{``A statistical test for conserved
  RNA structure shows lack of evidence for structure in lncRNAs''},
Nature Methods 14, 45–48 (2017).\\
\url{https://www.nature.com/articles/nmeth.4066}\\

\noindent
Rivas, E. and Eddy, S. E., \textit{``Response to Tavares et al.,
  “Covariation analysis with improved parameters reveals conservation
  in lncRNA structures”''}, (2018).\\
\url{https://doi.org/10.1101/2020.02.18.955047}.\\

\noindent
Rivas, E. \textit{et al.}, \textit{``Estimating the power of sequence
  covariation analysis for detecting conserved RNA structure''},
\textit{Bionformatics}, 36, 3072–3076, (2020).\\
\url{https://doi.org/10.1093/bioinformatics/btaa080}\\

\noindent
Rivas, E., \textit{``RNA structure prediction using
  positive and negative evolutionary information''},
\textit{PLOS Comput Biol}, 16(10), e1008387, (2020).\\
\url{https://doi.org/10.1371/journal.pcbi.1008387}.\\

\noindent
Karan, A and Rivas, E., \textit{``All-at-once RNA folding with 3D motif prediction framed by evolutionary information''},
\textit{bioRxiv}, Dec (2024).\\
\url{https://doi.org/}.\\






  









