
\section{Tutorial}
\label{section:tutorial}
\setcounter{footnote}{0}

Here's a tutorial walk-through of how to use \etwomsa. This should
suffice to get you started.

\subsection{Running \etwomsa}
To run \etwomsa\ with default parameters on an aligned file
\prog{tutorial/globins4.sto} or an unaligned file \prog{tutorial/globins4.fa} use:\\

\user{bin/e2msa tutorial/globins4.sto}\\
\user{bin/e2msa tutorial/globins4.fa}\\

\noindent
The output is an alignment, followed by some statistical information about the aligment.

\begin{sreoutput}
2msa-AFG alignment
# STOCKHOLM 1.0
#=GF ID HBB_HUMAN
#=GF AC HBB_HUMAN

HBB_HUMAN  ........VHLTPEEKSAVTALWGKVN--VDEVGGEALGRLLVVYPWTQRFFESF------GDLSTPDAVM------GNPKVKAHGKKVLGAFSDGLAHLDNLKGT
HBA_HUMAN  ........V-LSPADKTNVKAAWGKVGAHAGEYGAEALERMFLSFPTTKTYFPHF------------DLSH------GSAQVKGHGKKVADALTNAVAHVDDMPNA
MYG_PHYCA  ........-VLSEGEWQLVLHVWAKVEADVAGHGQDILIRLFKSHPETLEKFDRF------------KHLKTEAEMKASEDLKKHGVTVLTALGAILKKKGHHEAE
GLB5_PETMA pivdtgsvAPLSAAEKTKIRSAWAPVYSTYETSGVDILVKFFTSTPAAQEFFPKFKGLTTA------DQLK------KSADVRWHAERIINAVNDAVASMDDTEKM
//
name        %id     %match     alen    avg_indel_num    avg_indel_len    max_indel_len  ancestral_len      avd_seq_len
HBB_HUMAN    28%     81%       186     5.8 +/- 1.5     6.7 +/- 2.9      12           133                147.2 +/- 5.1 
\end{sreoutput}

A file  \prog{globins4.e2msa.AFG.sto} with the resulting alignment in Stochkholm format is also produced.

\subsection{Default parameters}

Default parameters are:

\begin{sreitems}{\emprog{Evolutionary model:}}
\item[\emprog{Evolutionary model:}] By default \etwomsa\ uses the AFG evolutionary model described in the paper.
  Based on our benckmarks, it is the evolutionary model that performs most favorably.
  Other alternative models that can be used are: AIF, AGA, AFGR, AALI, TKF91, and TKF92. All models are described
  in the \etwomsa\ publication.
  
\item[\emprog{Model parameters:}] The default parameters can be obtained using \user{\etwomsa\ -h}.
  We recommend that you use the trained parameters specific for each evolutionary model that have been
  obtained as described in the  \etwomsa\ publication, and provided with this package.

  We recommend to use:
  
  \user{bin/e2msa --paramfile data/training/Pfam.seed.S1000.trainGD.AFG.param tutorial/globins4.fa}\\

  Trained parameters for the following evolutionary models are also provided: AALI, AFGR, AGA, AIF, TKF92, TKF91.

\end{sreitems}

 A full list of the \etwomsa\ options is found by using

 \user{\etwomsa\ -h}

 
