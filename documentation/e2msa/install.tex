\section{Installation}
\label{section:installation}
\setcounter{footnote}{0}

\subsection{Quick installation instructions}

Download \prog{\etwomsa.tar.gz} from \url{http://eddylab.org/}; unpack
it, configure, and make:\\

\user{tar xf e2msa.tar.gz}\\
\user{cd e2msa}\\
\user{./configure}\\ 
\user{make}\\
\user{make install}\\

The newly compiled binary (\prog{\etwomsa}) is in the
\prog{\etwomsa/bin} directory. You can run it from there,
as in this example:\\

\user{bin/e2msa tutorial/globins4.sto}\\


That's it.  You can keep reading if you want to know more about
customizing a \etwomsa\ installation, or you can skip ahead to the next
chapter, the tutorial.


\subsection{System requirements}

\paragraph{Operating system:} \etwomsa\ is designed to run on
POSIX-compatible platforms, including UNIX, Linux and Mac OS/X. The
POSIX standard essentially includes all operating systems except
Microsoft Windows. We have tested most extensively on Linux and
MacOS/X because these are the machines we develop on.

\paragraph{Compiler:} The source code is C conforming to POSIX and ANSI
C99 standards. It should compile with any ANSI C99 compliant compiler,
including the GNU C compiler \prog{gcc}, and the C++ compiler
\prog{g++}. We test the code using the \prog{gcc} and \prog{g++}
compilers.

\subsection{Makefile targets}

\begin{sreitems}{\emprog{distclean}}

\item[\emprog{all}]
  Builds everything. Same as just saying \ccode{make}.

\item[\emprog{install}] 
  Installs the binaries (\prog{e2msa}, \prog{e2sim}, \prog{e2train}, \prog{e1sim},  \prog{gappenalties}, \prog{scorematrix2rate}).

  By default, programs are installed in
  \prog{e2msa\_version/bin}. 
  You can customize the location of the binaries by replacing
  
  \user{./configure} 
  
  with
  
  \user{./configure --prefix=/the/directory/you/want}
  
  The newly compiled binaries are now in the
  \prog{/the/directory/you/want/bin} directory.\\
  
\item[\emprog{clean}]
  Removes all files generated by compilation (by
  \ccode{make}). Configuration (files generated by
  \ccode{./configure}) is preserved.

\item[\emprog{distclean}]
  Removes all files generated by configuration (by \ccode{./configure})
  and by compilation (by \ccode{make}). 

\end{sreitems}


\subsection{What gets installed by 'make install', and where?}

The top-level configure file has a variable E2MSA\_HOME that
specifies the directory where \ccode{make install} will install
things: \ccode{E2MSA\_HOME/bin}.\\

By default E2MSA\_HOME is assigned to the current directory
e2msa.\\

The best way to change this default is when you use
\ccode{./configure}, and the most important variable to consider
changing is \ccode{--prefix}. For example, if you want to install
\etwomsa\ in a directory hierarchy all of its own, you might want to do
something like:

\user{./configure --prefix=/usr/local/e2msa}

That would keep \etwomsa\ out of your system-wide directories like
\ccode{/usr/local/bin}, which might be desirable. Of course, if you do
it that way, you'd also want to add \ccode{/usr/local/e2msa/bin} to
your \ccode{\$PATH}.
