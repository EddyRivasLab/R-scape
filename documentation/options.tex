\clearpage
\section{Options}
\label{section:options}
\setcounter{footnote}{0}

The whole list of options can be found using 

\user{R-scape -h}\\

Some important options are:
\subsection{Covariation statistic options}

\subsubsection{\prog{-E <x>}}  Target E-value is $x\geq 0$.

\subsubsection{\prog{--GT, --MI,  --MIr, --MIg, --CHI, --OMES, --RAF, --RAFS, }}
We favor the G-test covariation statistic, but a total of eight
covariation statistics are currently implemented in \rscape. For each
covariation statistic (GT, for instance), \rscape\ can also calculate
its average product correction (GTp) and its average sum corrections
(GTa). For each option above, appending ``p'' or ``a'' chooses one of the
corrections. For example, \prog{--GT} does the G-test statistic,
\prog{--GTp} does the APC-corrected G-test statistic, \prog{--GTa}
does the ASC-corrected G-test statistic.\\

The \rscape\ default is \prog{--GTp}.\\

Details of the definition and provenance of the different covariation
statistics can be found in the \rscape\ manuscript: Rivas, E. \& Eddy
S.~E., \textit{``A statistical test of RNA base pair covariation
  applied to proposed lncRNA structures''}. In a nutshell, given two
alignment columns $i,j$,
%
\[
\begin{array}{lrcl}
  \mbox{G-test:\citep{Woolf1957}}                                       & \mathrm{GT}(i,j)   & = & 2 \, \sum_{a,b} \mathrm{Obs}^{ab}_{ij} \, \log \frac{ \mathrm{Obs}^{ab}_{ij} } { \mathrm{Exp}^{ab}_{ij} }, \\
  \mbox{Pearson's chi-square:}                                          & \mathrm{CHI}(i,j)  & = &      \sum_{a,b} \frac{ \left(\mathrm{Obs}^{ab}_{ij} - \mathrm{Exp}^{ab}_{ij}\right)^2 }{\mathrm{Exp}^{ab}_{ij}},\\ 
  \mbox{Mutual information:\citep{Shannon48,Gutell94b}}                 & \mathrm{MI}(i,j)   & = &      \sum_{a,b} P^{ab}_{ij} \, \log \frac{ P^{ab}_{ij} }{ p^{a}_{i} \, p^{b}_{j}},                       \\
  \mbox{MI normalized:\citep{Martin05}}                                 & \mathrm{MIr}(i,j)  & = & \frac{\mathrm{MI}(i,j)} {H(i,j)} = \frac{\mathrm{MI}(i,j)} { -\sum_{a,b} P^{ab}_{ij} \log  P^{ab}_{ij}},        \\
  \mbox{MI with gap penalty:\citep{LindgreenKrogh06}}                   & \mathrm{MIg}(i,j)  & = & \mathrm{MI}(i,j) - \frac{N^G_{ij}} {N},                                                                   \\
  \mbox{Obs-Minus-Exp-Squared:\citep{Fodor04}}                          & \mathrm{OMES}(i,j) & = &      \sum_{a,b} \frac{ \left(\mathrm{Obs}^{ab}_{ij} - \mathrm{Exp}^{ab}_{ij}\right)^2 }{N_{ij}},                \\
  \mbox{RNAalifold (RAF):\citep{Hofacker02}}                            & \mathrm{RAF}(i,j)  & = & \mathrm{B}_{i,j},                                      \\
  \mbox{RNAalifold Stacking (RAFS):\citep{LindgreenKrogh06}}            & \mathrm{RAFS}(i,j) & = & \frac{1}{4}\left(\mathrm{B}_{i-1,j+1}+2\,\mathrm{B}_{i,j}+\mathrm{B}_{i+1,j-1}\right).                                      \\
\end{array}
\]
%
\noindent
where $a,b$ are (non-gap) residues; $N$ is the total number of aligned
sequences; $\mathrm{Obs}^{ab}_{ij}$ is the observed count of $a:b$
pairs in columns $i,j$ (only counting when both a,b are residues);
$N_{ij}$ is the total number of residue pairs in columns $i,j$ (only
counting when both a,b are residues); $P^{ab}_{ij}$ is the observed
frequency of pair $a:b$ in columns $i,j$
($P^{ab}_{ij}=\frac{Obs^{ab}_{ij}}{N_{ij}}$); $\mathrm{Exp}^{ab}_{ij}=
N_{ij} p^a_ip^b_j$ is the expected frequency of pair $a:b$ assuming
$i,j$ are independent, where $p^a_i$ are the marginal frequencies of
$a$ residues in column $i$ (averaged to all other positions) ($p^a_i
= \frac{1}{L-1}\sum_{j\neq i} \sum_b P^{ab}_{ij}$); $N^G_{ij} = N -
N_{ij}$ is the number of pairs involving at least one gap symbol; the
definition of $\mathrm{B}_{i,j}$ used in the RAF and RAFS statistics
is involved, a concise definition can be found
elsewhere~\citep{LindgreenKrogh06}.

The background corrections~\citep{DunnGloor07} for a given
covariation statistic above $\mathrm{COV}(i,j)$ are,
%
\[
\begin{array}{lrcl}
  \mbox{\small Average product correction} & \mathrm{COVp}(i,j) & = &  \mathrm{COV}(i,j) - \frac { \mathrm{COV}(i) \mathrm{COV}(j) } { \mathrm{COV} }, \\
  \mbox{\small Average sum correction}     & \mathrm{COVa}(i,j) & = &  \mathrm{COV}(i,j) - \left( \mathrm{COV}(i) + \mathrm{COV}(j) - \mathrm{COV} \right). \\
\end{array}
\]



\subsubsection{\prog{--C2, --C16}}
For all the covariation statistics (except RAF and RAFS), one can do a
16-component (C16) or a two-component (C2) calculation, depending on
whether it uses the 16 possible pair combinations, or those are group
in two classes depending on whether they form a Watson-Crick pair (6
cases, including U:G and G:U), or whether they do not (10 cases).

\rscape's default is the 16 component covariation statistic, unless
the number of sequences in the alignment is $\leq$ 8 or the length of
the alignment is $\leq$ 50, in which case it uses the two-class
covariation statistic.

\subsection{Search options}

\subsubsection{\prog{--cyk}} An optimal secondary structure is computed that includes all significant base pairs. The files for this
maximum-covariation optimal structure all include the suffix
\prog{.cyk.}


\subsubsection{\prog{--window <n>}} \rscape\ can be run in a window scanning version for long alignments.
The window size is $n>0$.

\subsubsection{\prog{--slide <n>}} In scanning mode, this options sets the number of positions to move from window to window, $n >0$.


\subsection{Input alignment options}

\subsubsection{\prog{-I <x>}} Only sequences with less than $0<x\leq 1$
pairwise similarity are considered in the analysis.  Sequences with at
least $x$ pairwise similarity are clustered together, and one
representative of the cluster is randomly selected. If this option is
not used all sequences are used in the analysis.

\subsubsection{\prog{--gapthresh <x>}} Only columns with less than $0<x\leq 1$ fraction of gaps are considered in the analysis.

\subsubsection{\prog{--submsa <n>}} Analyzes a random subset of the input alignment.

\subsubsection{\prog{--treefile <f>}} A phylogenetic tree in Newick format can be given (by default a tree is created 
from the alignment using the program FastTree~\citep{Price10}).  \rscape\ checks that the  number of taxa and the names
of the taxa matches for all alignments analyzed.


\subsection{Output options}

\subsubsection{\prog{--roc }} Tabular output that provides statistics for 
each score value.

File \emprog{tutorial/updated\_Arisong.roc} looks like:

\user{more tutorial/updated\_Arisong.roc}
\begin{sreoutput}
# MSA nseq 95 alen 65 avgid 66.352419 nbpairs 20 (20)
# Method: GTp
#cov_score  FP  TP Found  True  Negatives  Sen   PPV     F       E-value
121.79543   0   2  2      20    2060       10.00 100.00  18.18   4.07104e-05
121.44018   0   2  2      20    2060       10.00 100.00  18.18   4.29443e-05
121.08494   0   2  2      20    2060       10.00 100.00  18.18   4.53006e-05
120.72970   0   2  2      20    2060       10.00 100.00  18.18   4.53006e-05
...
\end{sreoutput}

This file produces a tabular output for each alignment as a function
of the covariation score, for plotting ROC curves. The values in the
file are described by the comment line. Notice that the number of
Trues (column 5) and Negatives (column 6) are fixed for a given
secondary structure and do not change.

\subsubsection{\prog{--outmsa <f>}} The actual alignment analyzed can be saved in Stockholm format to file $<$f$>$.


\subsection{Plotting options}

\subsubsection{\prog{--nofigures}} None of the graphical outputs are produced using this option.


\subsubsection{\prog{--r2rall}} Forces R2R to draw all positions in the alignment. By default only
those that are more than 50\% occupied or are base paired are
depicted.


\subsection{Other options}

\subsubsection{\prog{--seed <n>}} Sets the seed of the random number generator to $<$n$>$. Use n = 0 for a random seed.










