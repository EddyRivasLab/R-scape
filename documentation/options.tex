
\section{Options}
\label{section:options}
\setcounter{footnote}{0}

The whole list of options can be found using 

\user{R-scape -h}.\\

Some important options are:
\subsection{Covariation statistic options}

\subsubsection{\prog{-E <x>}}

\subsubsection{\prog{--GT, --MI,  --MIr, --MIg, --CHI, --OMES, --RAF, --RAFS, }}
We favor the G-test covariation statistic, but a total of eight
covariation statistics are currently implemented in \rscape. For each
covariation statistic (GT, for instance), \rscape\ can also calculate
its average product correction (GTp) ad its average sum corrections
(GTa).  Each option \prog{--GT} stands for three independent ones:
\prog{--GT, --GTp, --GTa}.

The \rscape\ default is \prog{--GTp}.

Details of the definition and provenance of the different covariation
statistics can be found in the \rscape\ manuscript: Rivas, E. \& Eddy
S.~E., \textit{``A statistical test of RNA base pair covariation
  applied to proposed lncRNA structures}.

\subsubsection{\prog{--C2, --C16}}
For all the covariation statistics (except RAF and RAFS), one can do a
16-component (C16) or a two-component (C2) calculation, depending on
whether it uses the 16 possible pair combinations, or those are group
in two classes depending on whether they form a Watson-Crick pair (6
cases, including U:G and G:U), or whether they do not (10 cases).

\rscape's default is the 16 component covariation statistic, unless
the number of sequences in the alignment is $\leq$ 8 or the length of
the alignment is $\leq$ 50, in which case it uses the two-class
covariation statistic.

\subsection{Search options}

\subsubsection{\prog{--cyk}}


\subsubsection{\prog{--window <n>}}

\subsubsection{\prog{--slide <n>}}

\subsection{Input alignment options}

\subsubsection{\prog{-I <x>}}

\subsubsection{\prog{--gapthresh <x>}}

\subsubsection{\prog{--submsa <n>}}

\subsubsection{\prog{--tree <s>}}


\subsection{Output options}

\subsubsection{\prog{--outmsa <f>}}


\subsection{Plotting options}

\subsubsection{\prog{--nofigures}}


\subsubsection{\prog{--r2rall}}













