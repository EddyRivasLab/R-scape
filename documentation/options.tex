\clearpage
\section{Options}
\label{section:options}
\setcounter{footnote}{0}

The whole list of options can be found using 

\user{R-scape -h}\\

Some important options are:
\subsection{Covariation statistic options}

\subsubsection{\prog{-E <x>}}  Target E-value is $x\geq 0$.

\subsubsection{\prog{--GT, --MI,  --MIr, --MIg, --CHI, --OMES, --RAF, --RAFS, }}
We favor the G-test covariation statistic, but a total of eight
covariation statistics are currently implemented in \rscape. For each
covariation statistic (GT, for instance), \rscape\ can also calculate
its average product correction (GTp) and its average sum corrections
(GTa). For each option above, appending ``p'' or ``a'' chooses one of the
corrections. For example, \prog{--GT} does the G-test statistic,
\prog{--GTp} does the APC-corrected G-test statistic, \prog{--GTa}
does the ASC-corrected G-test statistic.\\

The \rscape\ default is \prog{--GTp}.\\

Details of the definition and provenance of the different covariation
statistics can be found in the \rscape\ manuscript: Rivas, E. \& Eddy
S.~E., \textit{``A statistical test for conserved RNA structure shows
lack of evidence for structure in lncRNAs''}.

\noindent
In a nutshell, given two
alignment columns $i,j$,
%
\[
\begin{array}{lrcl}
  \mbox{G-test:\citep{Woolf1957}}                                       & \mathrm{GT}(i,j)   & = & 2 \, \sum_{a,b} \mathrm{Obs}^{ab}_{ij} \, \log \frac{ \mathrm{Obs}^{ab}_{ij} } { \mathrm{Exp}^{ab}_{ij} }, \\
  \mbox{Pearson's chi-square:}                                          & \mathrm{CHI}(i,j)  & = &      \sum_{a,b} \frac{ \left(\mathrm{Obs}^{ab}_{ij} - \mathrm{Exp}^{ab}_{ij}\right)^2 }{\mathrm{Exp}^{ab}_{ij}},\\ 
  \mbox{Mutual information:\citep{Shannon48,Gutell94b}}                 & \mathrm{MI}(i,j)   & = &      \sum_{a,b} P^{ab}_{ij} \, \log \frac{ P^{ab}_{ij} }{ p^{a}_{i} \, p^{b}_{j}},                       \\
  \mbox{MI normalized:\citep{Martin05}}                                 & \mathrm{MIr}(i,j)  & = & \frac{\mathrm{MI}(i,j)} {H(i,j)} = \frac{\mathrm{MI}(i,j)} { -\sum_{a,b} P^{ab}_{ij} \log  P^{ab}_{ij}},        \\
  \mbox{MI with gap penalty:\citep{LindgreenKrogh06}}                   & \mathrm{MIg}(i,j)  & = & \mathrm{MI}(i,j) - \frac{N^G_{ij}} {N},                                                                   \\
  \mbox{Obs-Minus-Exp-Squared:\citep{Fodor04}}                          & \mathrm{OMES}(i,j) & = &      \sum_{a,b} \frac{ \left(\mathrm{Obs}^{ab}_{ij} - \mathrm{Exp}^{ab}_{ij}\right)^2 }{N_{ij}},                \\
  \mbox{RNAalifold (RAF):\citep{Hofacker02}}                            & \mathrm{RAF}(i,j)  & = & \mathrm{B}_{i,j},                                      \\
  \mbox{RNAalifold Stacking (RAFS):\citep{LindgreenKrogh06}}            & \mathrm{RAFS}(i,j) & = & \frac{1}{4}\left(\mathrm{B}_{i-1,j+1}+2\,\mathrm{B}_{i,j}+\mathrm{B}_{i+1,j-1}\right).                                      \\
\end{array}
\]
%
\noindent
where $a,b$ are (non-gap) residues; $N$ is the total number of aligned
sequences; $\mathrm{Obs}^{ab}_{ij}$ is the observed count of $a:b$
pairs in columns $i,j$ (only counting when both a,b are residues);
$N_{ij}$ is the total number of residue pairs in columns $i,j$ (only
counting when both a,b are residues); $P^{ab}_{ij}$ is the observed
frequency of pair $a:b$ in columns $i,j$
($P^{ab}_{ij}=\frac{Obs^{ab}_{ij}}{N_{ij}}$); $\mathrm{Exp}^{ab}_{ij}=
N_{ij} p^a_ip^b_j$ is the expected frequency of pair $a:b$ assuming
$i,j$ are independent, where $p^a_i$ are the marginal frequencies of
$a$ residues in column $i$ (averaged to all other positions) ($p^a_i
= \frac{1}{L-1}\sum_{j\neq i} \sum_b P^{ab}_{ij}$); $N^G_{ij} = N -
N_{ij}$ is the number of pairs involving at least one gap symbol; the
definition of $\mathrm{B}_{i,j}$ used in the RAF and RAFS statistics
is involved, a concise definition can be found
elsewhere~\citep{LindgreenKrogh06}.

The background corrections~\citep{DunnGloor07} for a given
covariation statistic above $\mathrm{COV}(i,j)$ are,
%
\[
\begin{array}{lrcl}
  \mbox{\small Average product correction} & \mathrm{COVp}(i,j) & = &  \mathrm{COV}(i,j) - \frac { \mathrm{COV}(i) \mathrm{COV}(j) } { \mathrm{COV} }, \\
  \mbox{\small Average sum correction}     & \mathrm{COVa}(i,j) & = &  \mathrm{COV}(i,j) - \left( \mathrm{COV}(i) + \mathrm{COV}(j) - \mathrm{COV} \right). \\
\end{array}
\]



\subsubsection{\prog{--C2, --C16}}
For all the covariation statistics (except RAF and RAFS), one can do a
16-component (C16) or a two-component (C2) calculation, depending on
whether it uses the 16 possible pair combinations, or those are group
in two classes depending on whether they form a Watson-Crick pair (6
cases, including U:G and G:U), or whether they do not (10 cases).

\rscape's default is the 16 component covariation statistic, unless
the number of sequences in the alignment is $\leq$ 8 or the length of
the alignment is $\leq$ 50, in which case it uses the two-class
covariation statistic.

\subsection{Search options}


\subsubsection{\prog{-s}} The ``two-set test'' option.
This option requires that a structure is provided with the alignment.
If option \prog{-s} is used, \rscape\ performs two independent test,
one for the given structure, another for all other possible pairs.
The default is a ``one-set test'' in which all possible pairs in the
alignment are tested equivalently.


\subsubsection{\prog{--fold}} A CaCoFold structure is computed that includes all significant base pairs. All files related to this 
CaCoFold structure include the suffix \prog{.fold.}

When option \prog{--fold} is used, a file with the original alignment  annotated with the R-scape structure in Stockholm format is produced.
This alignment has the suffix \prog{.fold.sto}.

\subsubsection{\prog{--naive}} Reports the laundry list of all covariation scores, without any statistical significance (E-value)
associated to them. No null alignments are created.

\subsubsection{\prog{--tstart <n>}} Analyze starting from position $n >= 1$ in the alignment.

\subsubsection{\prog{--tend <n>}} Analyze ending at position $n <= L$ in the alignment.

\subsubsection{\prog{--window <n>}} \rscape\ can be run in a window scanning version for long alignments.
The window size is $n>0$.

\subsubsection{\prog{--slide <n>}} In scanning mode, this options sets the number of positions to move from window to window, $n >0$.

\subsubsection{\prog{--vshuffle}} Vertical shuffle, a developers tool. Before performing any analysis, it shuffles all residues in each alignment column independently.

\subsubsection{\prog{--cshuffle}} Column shuffle, a developers tool. Before performing any analysis, it shuffles all columns in the alignment.

\subsubsection{\prog{--givennull <f>}} Use histrogram provided in file $<f>$ as null. 



\subsection{Input alignment options}

\subsubsection{\prog{-I <x>}} Only sequences with less than $0<x\leq 1$
pairwise similarity are considered in the analysis.  Pairwise \%
identity is defined as the ratio of identical positions divided by the
minimum length of the two sequences. If this option is not used all
(weighted) sequences are used in the analysis.

\subsubsection{\prog{--gapthresh <x>}} Only columns with less than $0<x\leq 1$ fraction of gaps are considered in the analysis.

\subsubsection{\prog{--consensus}} If the alignment has a GC ``seq\_cons'' field, only consensus positions will be analyzed.

\subsubsection{\prog{--submsa <n>}} Analyzes a random subset of the input alignment.

\subsubsection{\prog{--treefile <f>}} A phylogenetic tree in Newick format can be given (by default a tree is created 
from the alignment using the program FastTree~\citep{Price10}).  \rscape\ checks that the  number of taxa and the names
of the taxa matches for all alignments analyzed.


\subsection{Options for producing a CaCoFold structure}

\subsubsection{\prog{--fold}}  
When using the option \prog{--fold}, R-scape engages the CaCoFold
algorithm to produce a predicted structure. The CaCoFold algorithm
incorporates all positive (significantly covarying) basepairs, and
prevents any negative pair (pairs that have power of covariation but
not covariation) from happening. The CaCoFold algorithm uses a
recursive cascade of constrained foldings. The first fold uses the RBG probabilistic grammar, the rest use the G6X probabilistic grammar.

\noindent
Regarding the predicted structure, CaCoFold can use one two algorithms:

\subsubsection{\prog{--cyk}}        Default option. Each folding reports the structure with the best probability using the CYK algorithm.

\subsubsection{\prog{--decoding}}   This options returns the structure obtained by posterior decoding.

\noindent
Posterior decoding usually performs better than CYK. Both algorithm has the same algorithmic complexity. CYK is faster.

\noindent
Several additional options can be used in combination with   \prog{--fold},

\subsubsection{\prog{--refseq}}   By default the CaCoFold algorithm folds a profile sequence built from the alignment. Using this option, the sequence to fold is a consensus reference sequence.

\subsubsection{\prog{--E\_neg <x>}}   Pairs with E-value larger than the E-value cutoff but smaller than <x> will not be called negatives regardless of their covariation power. Default for E\_neg is 1.0.

\subsubsection{\prog{--lastfold}} This option forces one last alternative fold (using grammar G6X) after all covarying basepairs have already been
integrated into the structure. By default this last fold is not
performed. In the absence of any covarying basepair, one fold is
performed using grammar RBG.

\subsubsection{\prog{--show\_hoverlap}} This option leaves the alternative helices unmodified. By default, alternative structures are trimmed down to show no overlap with helices from the previous layers.

\subsubsection{\prog{--covmin <n>}} Minimum distance between position to report significant covariations. Default is 1, which means that significant covariations between
contiguous positions are reported.

\subsubsection{\prog{--allow\_negatives}} This option (just for developers) allows all basepairs to form regardless of their power.

\subsection{Options for importing a structure}

\rscape\ does not require to input a structure (either a RNA structure
or a protein contact map). By default \rscape\ analyzes all possible
pairs in the alignment.
\vspace{1mm}

\noindent
There are two ways to provide a contact map (or structure):

\begin{itemize}

\item By providing the alignment in Stockholm format with a ``ss\_cons'' field including the consensus
structure for the alignment. (For RNA alignments only.)

\item By analyzing a 3D structure provided in a PDB file. (For either RNA or peptide alignments.)\\
\end{itemize}

\noindent
These two methods can be combined together. For a nucleotide
alignment, if both a consensus structure is present in the alignment,
and a PDB file is provided (using option \prog{--pdb}), the
consensus structure will be extended by the information provided by
the pdbfile.  To ignore the consensus structure use
option \prog{--onlypdb}.\\


\noindent
From the PDB file we obtain three types of structural pairs:

\begin{itemize}
\item \textbf{Contacts:} defined as those two residues at a close spatial distance (specified by the user with option \prog{--cntmaxD}).
\item \textbf{Basepair:} RNA basepairs. \\
RNA basepairs are calculated using the program \prog{rnaview}~\citep{YangWesthof03}.

These RNA basepairs can be further classified in two types:
\begin{itemize}
\item \textbf{Watson-Crick basepairs:} the canonical RNA basepairs. mostly A:U, G:C, or G:U pairs. (H-bond interactions between two W-C faces in cis).
\item \textbf{Other basepairs:} the non-canonical RNA basepairs (all other types of H-bond interactions, 12 different types).
\end{itemize}
\end{itemize}

Contacts and RNA basepairs are extracted as follows:
\begin{itemize}

\item The spatial distance between any two residues is calculated as the  minimal Euclidean distance between any two atoms (excluding
 H atoms). Any two pairs at a distance not larger than a maximum value
(\texttt{contmaxD}) are called a ``contact''.

\item RNA basepairs are obtained using the program \texttt{rnaview}~\citep{YangWesthof03}\\
(http://ndbserver.rutgers.edu/ndbmodule/services/download/rnaview.html).\\
The RNA basepair annotation takes precedent over the annotation as ``contact''.
\end{itemize}

\vspace{2mm}
\noindent
The options that control the input of a structure or contact map are:

\subsubsection{\prog{--pdb <s>}} Reads a pdbfile associated to the alignment, and extracts the contacts
from it.

A ``.cmap'' file is produced reporting the structure obtained from the PDB file.

Option \prog{--pdb} is incompatible with \prog{--fold}.

\subsubsection{\prog{--cntmaxD <x>}} Maximum distance (in Angstroms) allowed between two residues to define a ``contact'' is $\langle x\rangle$.

\subsubsection{\prog{--cntmind <n>}} Minimum distance (in residue positions) in the backbone between two residues required to define a ``contact'' is $\langle n\rangle$.
\vspace{5mm}

\subsubsection{\prog{--onlypdb}} Reads the structure from the pdbfile and ignores the alignment consensus structure (if provided).

\subsubsection{\prog{--draw\_nonWC}} Adds the non-canonical basepairs into the structure graphical output. For clarity, the default is to draw only the Watson-Crick basepairs.
This option affects only the drawing of the structure. All basepairs (canonical or not) are used as part of the structure to perform the two-set statistical test.
\\\\

\noindent
Example of reading a structure from a PDB file for the FMN riboswitch:

\user{bin/R-scape --cntmaxD 4 --cntmind 3 --pdb tutorial/3f2q.pdb -s --onlypdb tutorial/RF00050.sto}\\

\noindent
This command line extracts contacts from the pdb file that are at a
Euclidean distance $\leq 4 \AA$ in the PDB structure, and such that
they are at least 3 residues apart in the backbone.


\noindent
The output is

\begin{sreoutput}
# R-scape :: RNA Structural Covariation Above Phylogenetic Expectation
# R-scape 0.8.1 (Jul 2018)
# Copyright (C) 2016 Howard Hughes Medical Institute.
# Freely distributed under the GNU General Public License (GPLv3).
# - - - - - - - - - - - - - - - - - - - - - - - - - - - - - - - - - - - -
# Two-set statistical test (one test for annotated basepairs, another for all other pairs)
#
# Structure obtained from the pdbfile
# ij in alignment | ij in pdbsequence | basepair type
# 3 218 | 1 112 | WWc
# 4 216 | 2 110 | CONTACT
# 4 217 | 2 111 | WWc
# 4 218 | 2 112 | CONTACT
# 5 216 | 3 110 | WWc
# 5 217 | 3 111 | CONTACT
# 6 215 | 4 109 | WWc
# 6 216 | 4 110 | CONTACT
# .
# .
# .
# 192 202 | 87 96 | WWc
# 192 203 | 87 97 | CONTACT
# 193 198 | 88 92 | CONTACT
# 193 201 | 88 95 | WWc
# 193 202 | 88 96 | CONTACT
# 195 197 | 89 91 | CONTACT
# 195 198 | 89 92 | WHt
# 198 200 | 92 94 | CONTACT
# 198 201 | 92 95 | CONTACT
# 205 207 | 99 101 | CONTACT
# PDB:      versions/rscape/rscape_v0.8/tutorial/3f2q.pdb
# contacts  169 (49 bpairs 35 wc bpairs)
# maxD      4.00
# mind      3
# distance  MIN
# L         139
# alen      221
# pdblen    112
# ::[[[[[[[[,,,,,,<<<_______>>>,((((<<<<_________________AA>>>>,,<<<----------<____>>>>,,,<<<<<_______>>>>>aa))AAAA----))aaaa,,,,,,]]]]]]]]::
# MSA RF00050_FMN.3f2q nseq 144 (144) alen 139 (221) avgid 69.18 (68.15) nbpairs 49 (0)
#
# Method Target_E-val [cov_min,conv_max] [FP | TP True Found | Sen PPV F] 
# GTp    0.05         [-9.78,216.11]     [1 | 14 49 15 | 28.57 93.33 43.75] 
#
#       left_pos       right_pos        score           E-value
#------------------------------------------------------------------
*	     171	     183	216.11095	1.6421e-10
*	     170	     184	211.69081	2.76699e-10
*	     192	     202	168.72417	4.95548e-08
*	       8	     213	149.71776	4.89982e-07
*	     172	     182	138.66664	1.84675e-06
*	     169	     185	137.23189	2.21548e-06
**	      16	      30	133.44999	3.53772e-06
*	       5	     216	131.02575	4.70876e-06
*	      84	     186	125.60806	9.0169e-06
*	      17	      29	112.04610	4.62895e-05
*	       7	     214	111.12654	5.13519e-05
*	       6	     215	96.43781	0.00029929
*	      36	      87	96.32752	0.00029929
*	      94	     163	78.81578	0.0024303
 	       7	     213	107.68588	0.0147937
\end{sreoutput}

\noindent
All coordinates are relative to the input alignment. The annotation of
all types of RNA basepairs (WWc, WWt, WHc,...) is produced by the
program \prog{rnaview}~\citep{YangWesthof03}.

\subsection{Options for type of pairs tested}

When performing the two-class statistical test (option \prog{-s})
using a pdbfile to read the structure, there are different options as
to which types of basepairs are used to define the sample size for the
basepairs test.

The options are:

\subsubsection{\prog{--samplecontacts}}
The basepair statistical test includes all the contacts identified in a PDB
or/and as a RNA secondary structure included with a input alignment in
Stockholm format.  This is the default option for amino acid
alignments if a PDB file is provided.  

\subsubsection{\prog{--samplebp}} For RNA alignments with only.
The basepair statistical test includes basepairs of all 12 possible types.
This is the default option for RNA/DNA alignments if a PDB file is
provided.  

\subsubsection{\prog{--samplewc}} For RNA alignments only.
The basepair statistical test includes only the canonical
(Watson-Crick/Watson-Crick type) basepairs (A:U, G:C, G:U).  This is
the default option for RNA/DNA alignments if a consensus secondary
structure is provided. 

\subsection{Output options}

\subsubsection{\prog{--roc}}

Produces a tabular output that provides statistics for each score value.

File \emprog{tutorial/updated\_Arisong.roc} looks like:

\user{more tutorial/updated\_Arisong.roc}
\begin{sreoutput}
# MSA nseq 95 alen 65 avgid 66.352419 nbpairs 20 (20)
# Method: GTp
#cov_score  FP  TP Found  True  Negatives  Sen   PPV     F       E-value
121.79543   0   2  2      20    2060       10.00 100.00  18.18   4.07104e-05
121.44018   0   2  2      20    2060       10.00 100.00  18.18   4.29443e-05
121.08494   0   2  2      20    2060       10.00 100.00  18.18   4.53006e-05
120.72970   0   2  2      20    2060       10.00 100.00  18.18   4.53006e-05
...
\end{sreoutput}

This file produces a tabular output for each alignment as a function
of the covariation score, for plotting ROC curves. The values in the
file are described by the comment line. Notice that the number of
Trues (column 5) and Negatives (column 6) are fixed for a given
secondary structure and do not change.

\subsubsection{\prog{--outmsa <f>}} The actual alignment analyzed can be saved in Stockholm format to file $<$f$>$.

\subsubsection{\prog{--outtree <f>}} The phylogenetic tree (created using the program FastTree) can be saved in Newick format to file $<$f$>$.

\subsubsection{\prog{--savenull}} Saves a histogram with the null distribution to file rnafile msaname.null.

\subsection{Plotting options}

\subsubsection{\prog{--nofigures}} None of the graphical outputs are produced using this option.


\subsubsection{\prog{--r2rall}} Forces R2R to draw all positions in the alignment. By default only
those that are more than 50\% occupied or are base paired are
depicted.


\subsection{Other options}

\subsubsection{\prog{--seed <n>}} Sets the seed of the random number generator to $<$n$>$. Use n = 0 for a random seed.










