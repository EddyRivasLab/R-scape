
\section{Some other topics}
\label{section:more}
\setcounter{footnote}{0}

\subsection{How do I cite \rscape?}

Pending a publication, the appropriate citation is to the web server,
\url{github.com/EddyRivasLab/R-scape}.

You should also cite what version of the software you used. We archive
all old versions, so anyone should be able to obtain the version you
used, when exact reproducibility of an analysis is an issue.

The version number is in the header of most output files. To see it
quickly, do something like \prog{R-scape -h} to get a help page, and
the header will say:

\begin{sreoutput}
# R-scape :: RNA Structural Covariation Above Phylogenetic Expectation
# R-scape 0.2 (June 2016)
# Copyright (C) 2016 Howard Hughes Medical Institute.
# Freely distributed under the GNU General Public License (GPLv3).
# - - - - - - - - - - - - - - - - - - - - - - - - - - - - - - - - - - - -
\end{sreoutput}

So (from the second line there) this is from \rscape\ v0.1.

\subsection{How do I report a bug?}

Email us, at \url{elenarivas@fas.harvard.edu}.

Before we can see what needs fixing, we almost always need to
reproduce a bug on one of our machines. This means we want to have a
small, reproducible test case that shows us the failure you're seeing.
So if you're reporting a bug, please send us:

\begin{itemize}
 \item A brief description of what went wrong.
 \item The command line(s) that reproduce the problem.
 \item Copies of any files we need to run those command lines.
 \item Information about what kind of hardware you're on, what
   operating system, and what compiler and version you used, with what
   configuration arguments.
\end{itemize}



