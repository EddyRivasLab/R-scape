\section{Installation}
\label{section:installation}
\setcounter{footnote}{0}

\subsection{Quick installation instructions}

Download \prog{\rscape.tar.gz} from \url{http://eddylab.org/}; unpack
it, configure, and make:\\

\user{tar xf R-scape.tar.gz}\\
\user{cd R-scape}\\
\user{./configure}\\ 
\user{make}\\
\user{make install}\\

The newly compiled binary (\prog{\rscape}) is in the
\prog{\rscape/bin} directory. You can run it from there,
as in this example:\\

\user{bin/R-scape tutorial/updated\_Arisong.sto}\\


That's it.  You can keep reading if you want to know more about
customizing a \rscape\ installation, or you can skip ahead to the next
chapter, the tutorial.


\subsection{System requirements}

\paragraph{Operating system:} \rscape\ is designed to run on
POSIX-compatible platforms, including UNIX, Linux and Mac OS/X. The
POSIX standard essentially includes all operating systems except
Microsoft Windows. We have tested most extensively on Linux and
MacOS/X because these are the machines we develop on.

\paragraph{Compiler:} The source code is C conforming to POSIX and ANSI
C99 standards. It should compile with any ANSI C99 compliant compiler,
including the GNU C compiler \prog{gcc}, and the C++ compiler
\prog{g++}. We test the code using the \prog{gcc} and \prog{g++}
compilers.

The code include several Perl scripts (from the independent program
R2R used here). Make sure your PATH environmental variable includes a
directory with a Perl executable.

\paragraph{Libraries and other installation requirements:}

\rscape\ includes two software libraries:

\begin{itemize}
\item the Easel library package (\url{http://bioeasel.org/}),
\item the HMMER library package (\url{http://hmmer.org/}),
\end{itemize}

and three independent programs:

\begin{itemize}
\item FastTree~\citep{Price10} (for building phylogenetic trees),
  
\item R2R~\citep{WeinbergBreaker11} (for drawing consensus RNA
  structures),

\item RNAVIEW~\citep{YangWesthof03} (for identifying different types of
  basepairs in nucleic acid alignments).
  
\end{itemize}

 All libraries and independent programs will automatically compile
 during \rscape's installation process.  By default, \rscape\ does not
 require any additional libraries to be installed by you, other than
 standard ANSI C99 libraries that should already be present on a
 system that can compile C code.

 Executables for the three independent programs will appear in the
 \prog{\rscape/bin} directory.

\subsection{Makefile targets}

\begin{sreitems}{\emprog{distclean}}

\item[\emprog{all}]
  Builds everything. Same as just saying \ccode{make}.

\item[\emprog{install}] 
  Installs the binaries (\prog{R-scape}, \prog{FastTree}, \prog{r2r}).

  By default, programs are installed in
  \prog{R-scape\_version/bin}. 
  You can customize the location of the binaries by replacing
  
  \user{./configure} 
  
  with
  
  \user{./configure --prefix=/the/directory/you/want}
  
  The newly compiled binaries are now in the
  \prog{/the/directory/you/want/bin} directory.\\
  
\item[\emprog{uninstall}]
  Reverses the steps of make install. 


\item[\emprog{clean}]
  Removes all files generated by compilation (by
  \ccode{make}). Configuration (files generated by
  \ccode{./configure}) is preserved.

\item[\emprog{distclean}]
  Removes all files generated by configuration (by \ccode{./configure})
  and by compilation (by \ccode{make}). 

\end{sreitems}

\subsection{Why is the output of 'make' so clean?}

Because we're hiding what's really going on with the compilation with
a wrapper.  If you want to see what the command lines really look
like, pass a \ccode{V=1} option (V for ``verbose'') to \ccode{make},
as in:

\user{make V=1}

\subsection{What gets installed by 'make install', and where?}

The top-level configure file has a variable RSCAPE\_HOME that
specifies the directory where \ccode{make install} will install
things: \ccode{RSCAPE\_HOME/bin}.\\

By default RSCAPE\_HOME is assigned to the current directory
R-scape.\\

The best way to change this default is when you use
\ccode{./configure}, and the most important variable to consider
changing is \ccode{--prefix}. For example, if you want to install
\rscape\ in a directory hierarchy all of its own, you might want to do
something like:

\user{./configure --prefix=/usr/local/rscape}

That would keep \rscape\ out of your system-wide directories like
\ccode{/usr/local/bin}, which might be desirable. Of course, if you do
it that way, you'd also want to add \ccode{/usr/local/rscape/bin} to
your \ccode{\$PATH}.
