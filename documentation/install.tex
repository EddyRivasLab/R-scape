\section{Installation}
\label{section:installation}
\setcounter{footnote}{0}

\subsection{Quick installation instructions}

Download \prog{\rscape\_version.tar.gz} from
\url{http://selab.org/}, or from
\url{hhps://hithub.com/EddyRivasLab/R-scape}; unpack it, configure,
and make:\\

\user{tar xf R-scape\_version.tar.gz}\\
\user{cd R-scape\_version}\\
\user{./configure}\\ 
\user{make}\\
\user{make install}\\

The newly compiled binary (\prog{\rscape}) is in the
\prog{\rscape\_version/bin} directory. You can run it from there.\\

That's it.  You can keep reading if you want to know more about
customizing a \rscape\ installation, or you can skip ahead to the next
chapter, the tutorial.


\subsection{System requirements}

\paragraph{Operating system:} \rscape\ is designed to run on
POSIX-compatible platforms, including UNIX, Linux and MacOS/X. The
POSIX standard essentially includes all operating systems except
Microsoft Windows. We have tested most extensively on MacOS/X because
these are the machines we develop on.

\paragraph{Compiler:} The source code is C conforming to POSIX and ANSI
C99 standards. It should compile with any ANSI C99 compliant compiler,
including the GNU C compiler \prog{gcc}. We test the code using both
the \prog{gcc} and \prog{icc} compilers. 

\paragraph{Libraries and other installation requirements:} \rscape\ includes
two software libraries: HMMER~\citep{Eddy11} and Easel (bundled as
part of HMMER), and two independent programs FastTree~\citep{Price10}
(for building phylogenetic trees) and R2R~\citep{WeinbergBreaker11}
(for drawing consensus RNA structures). All will automatically compile
during \rscape's installation process.  By default, \rscape\ does not
require any additional libraries to be installed by you, other than
standard ANSI C99 libraries that should already be present on a system
that can compile C code.

\subsection{Makefile targets}

\begin{sreitems}{\emprog{distclean}}

\item[\emprog{all}]
  Builds everything. Same as just saying \ccode{make}.

\item[\emprog{install}] 
  Installs the binaries (\prog{R-scape}, \prog{FastTree}, \prog{r2r}).

  By default, programs are installed in
  \prog{R-scape\_version/bin}. 
  You can customize the location of the binaries by replacing
  
  \user{./configure} 
  
  with
  
  \user{./configure --prefix=/the/directory/you/want}
  
  The newly compiled binaries are now in the
  \prog{/the/directory/you/want/bin} directory.\\
  
\item[\emprog{clean}]
  Removes all files generated by compilation (by
  \ccode{make}). Configuration (files generated by
  \ccode{./configure}) is preserved.

\item[\emprog{distclean}]
  Removes all files generated by configuration (by \ccode{./configure})
  and by compilation (by \ccode{make}). 

\end{sreitems}

