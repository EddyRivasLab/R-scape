\label{section:outputs}
\setcounter{footnote}{0}
\section{Outputs of \rscape}

For each alignment file \prog{rnafile.sto}, \rscape\, produces the following output:

\begin{sreitems}{\emprog{rnafile.sorted.out}}
\item[\emprog{rnafile.out}] Tabular output with the significant pairs,
  with their score and E-value.
%
\item[\emprog{rnafile.sorted.out}] Tabular output sorted from highest to
  lowest E-value.
%
\item[\emprog{rnafile.roc}] Tabular output that provides statistics for 
each score value.
%
\item[\emprog{rnafile.sum}] Tabular output with a line summary statistics
  per alignment in the file.
%
\end{sreitems}

An Stockholm alignment file can include several alignments.  For each
individual alignment (\prog{msaname}) in the Stockholm file,
\rscape\ produces the following output:

\subsection{Alignment with consensus seconary structure}
If the given alignment does have a consensus secondary structure
(\prog{\#=GF SS\_cons} markup), the following files are produced

\begin{sreitems}{\emprog{rnafile\_msaname.R2R.sto.\{pdf,svg\}}}
\item[\emprog{rnafile\_msaname.his}] A two column histogram file for the
  covariation scores.
%
\item[\emprog{rnafile\_msaname.his.ps}] Plot of the score histogram.
%
\item[\emprog{rnafile\_msaname.R2R.sto}] Stockholm file annotated by a
  modified version of the R2R program. This file includes the
  information necesary to draw the consensus structure, and to
  annotate the significantly covarying base pairs.
%
\item[\emprog{rnafile\_msaname.R2R.sto.\{pdf,svg\}}] Drawing of the
  \rscape-annotated consensus secondary structure.
%
\item[\emprog{rnafile\_msaname.dplot.\{ps,svg\}}] Dot plot of the consensus
  secondary structure annotated according to covariation.
%
\end{sreitems}
For each alignment, \emprog{msaname} is given
by \prog{<ACC>\_<ID>}, the combination of the accesion \prog{\#=GF
AC <AC>} and name \prog{\#=GF ID <ID>} stockholm-format markups (or
one of two if the other in not defined).  If none of those fields are
defined, \emprog{msaname} is a number describing the order in the
file of the given alignment.

\subsection{Alignment without consensus seconary structure}
Alternatively, if the alignemnt does not have a consensus secondary
structure (or if it does and the option \prog{\rscape\ --cyk} is
used) \rscape\, produces the following additional files describing the
maximal-covariation optimal secondary structure:

\begin{sreitems}{\emprog{rnafile\_msaname.cyk.R2R.sto.\{pdf,svg\}}}
\item[\emprog{rnafile\_msaname.cyk.his}]
%
\item[\emprog{rnafile\_msaname.cyk.his.\{ps.svg\}}]
%
\item[\emprog{rnafile\_msaname.cyk.R2R.sto}]
%
\item[\emprog{rnafile\_msaname.cyk.R2R.sto.\{pdf,svg\}}]
%
\item[\emprog{rnafile\_msaname.cyk.dplot.\{ps,svg\}}]
%
\end{sreitems}
These files are formatted identially to those for describing the given
consensus structure.



\subsection{Tabular outputs per file}

The \prog{tutorial/updated\_arisong.out} output file looks like this:

\begin{sreoutput}
more tutorial/updated_Arisong.out 
# MSA updated_Arisong_1 nseq 69 (95) alen 65 (150) avgid 65.16 (64.97) nbpairs 20 (20)
# GTp thresh Eval 0.050000 cov=41.310069 [-9.744726,89.076909] [2 | 9 20 11 | 45.000000 81.818182 58.064516] 
                93             104      43.65   6.01567e-06
*               94             110      43.17   4.29923e-05
*               96             108      65.95   0
*               98             106      89.08   0
...
\end{sreoutput}
The output is a list of significant pairs ordered by the first positions. \\

\begin{sreitems}{\emprog{Second and third columns}}
\item[\emprog{First column}] indicates whether the significant pair is
  part of the given structure (*), or not.  If the pair is not in the
  structure, we distinguish whether the pair is compatible with the
  given structure ($\sim$) o not, in which case it is a blank.

\item[\emprog{Second and third columns}] are the two positions of the
  pair. Positions are relative to the input alignment.

\item[\emprog{Forth column}] is the covariation score

\item[\emprog{Fifth column}] is the E-value. Significant positions
  have E-values $<< 1$.
\end{sreitems}

The first commented line on top describe properties of the given
alignment: number of sequence (nseq), alignment length (alen), average
percentage identity (avgid), and number of base pairs (nbpairs).
Values in parenthesis correspond to the alignment as is given. Values
not in parenthesis correspond to the analyzed alignment after the
default filters have been applied.

The second commented line describes the covariation method (GTp), the
E-value threshold (0.05), the score at that evalue (42.3), the range
of scores for all pairs in the alignmens (from -9.7 to 89.1), the
number of covaring not base pairs (2), the number of covarying base
pairs (9), the number of base pairs (20), and the total number of
covarying pairs (11). Lastly we provide the sensitivity (45.0=9/20),
positivie predictive value (81.8=9/11)), and F-measure (58.1 = 2 *
45.0 * 81.8 / (45.0+81.8))).\\

Other outputs are more relevant for benchmarking purposes, such as:\\
The \prog{tutorial/updated\_arisong.sum} file:

\begin{sreoutput}
more tutorial/updated_Arisong.sum 
0.050000        updated_Arisong_1       69      65      65.16    GTp 9 20 20 45.000000 45.000000 
\end{sreoutput}

The \prog{tutorial/updated\_arisong.roc} file:
\begin{sreoutput}
more tutorial/updated_Arisong.roc
# MSA nseq 69 alen 65 avgid 65.163163 nbpairs 20 (20)

# GTp thresh fp tf found true negatives sen ppv F evalue
89.04630 0 1 1 20 2060 5.00 100.00 9.52 0
88.79103 0 1 1 20 2060 5.00 100.00 9.52 0
88.53575 0 1 1 20 2060 5.00 100.00 9.52 0
88.28048 0 1 1 20 2060 5.00 100.00 9.52 0
88.02521 0 1 1 20 2060 5.00 100.00 9.52 0
87.76993 0 1 1 20 2060 5.00 100.00 9.52 0
87.51466 0 1 1 20 2060 5.00 100.00 9.52 0
...
\end{sreoutput}


\subsection{Graphical outputs per alignment}



